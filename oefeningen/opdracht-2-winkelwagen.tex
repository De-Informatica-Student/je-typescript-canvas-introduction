\documentclass{article}

\usepackage{hyperref}
\hypersetup{
	colorlinks=true,
	linkcolor=blue,
	filecolor=magenta,
	urlcolor=cyan
}

\title{TypeScript Oefeningen 2 - Winkelwagen}
\author{Wesley van Schaijk}
\date{2023-12-23}

\begin{document}

\maketitle

\paragraph{}
Een makkelijk beginpunt voor deze opdrachten is de \href{https://github.com/HZ-HBO-ICT/ts-skeleton-app}{TypeScript Skeleton repository} van school.

\begin{enumerate}
	\item Maak een webpagina met drie plaatjes van producten boven in het scherm op een rij met een hoogte van 200px, De styling komt in een los CSS-bestand.
	\item Plaats onder iedere afbeelding twee knoppen, een + en een -, zo kan de gebruiker dingen toevoegen aan de winkelwagen.
	\item Plaats onder deze drie elementen een lijst waarin staat hoeveel items zijn geselecteerd.
	\item Zorg ervoor dat de informatie in de lijst veranderd als de knoppen worden gebruikt.
	\item Zorg ervoor dat negatieve getallen niet kunnen.
	\item Verzin per product een willekeurige prijs, iedere keer dat de lijst wordt aangepast dient de totaalprijs opnieuw te worden berekend.
	\item Plaats het totaalbedrag aan de onderkant van de lijst.
\end{enumerate}

\end{document}
