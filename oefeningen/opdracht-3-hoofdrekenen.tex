\documentclass{article}

\usepackage{hyperref}
\hypersetup{
	colorlinks=true,
	linkcolor=blue,
	filecolor=magenta,
	urlcolor=cyan
}

\title{TypeScript Oefeningen 3 - Hoofdrekenen}
\author{Wesley van Schaijk}
\date{2023-12-25}

\begin{document}

\maketitle

\paragraph{}
Een makkelijk beginpunt voor deze opdrachten is de \href{https://github.com/HZ-HBO-ICT/ts-skeleton-app}{TypeScript Skeleton repository} van school.

\begin{enumerate}
	\item Maak een gecentreerde website met een titel ``Oefeningen Hoofdrekenen``, met daaronder een selectievakje en een knop ``Oefenen``.
	\item Vul het selectievakje met de opties `Optellen`, `Aftrekken`, `Vermenigvuldigen` en `Delen met Rest`.
	\item Op het moment dat de gebruiker op ``Oefenen`` klikt, dient er een lijst met sommen te verschijnen, 4 rijtjes van 5 sommen.
	\item De sommen zijn met natuurlijke (hele) getallen, geen komma getallen; de sommen worden willekeurig gegenereerd.
	\item De gebruiker moet op de website de sommen kunnen invullen, navigerend door middel van de tab-toets.
	\item Als de gebruiker alle sommen heeft ingevuld dient de tab toets de knop ``nakijken`` te selecteren.
	\item De tekstvakjes met goede antwoorden kleuren groen en met foute antwoorden kleuren rood.
	\item \textbf{Uitdaging:} Voeg een `Gecombineerd` optie toe die een test maakt met verschillende soorten sommen.
\end{enumerate}

\end{document}
