\documentclass{article}

\usepackage{hyperref}
\hypersetup{
	colorlinks=true,
	linkcolor=blue,
	filecolor=magenta,
	urlcolor=cyan
}

\title{TypeScript Oefeningen 4 - Verlanglijstje}
\author{Wesley van Schaijk}
\date{2023-12-26}

\begin{document}

\maketitle

\paragraph{}
Een makkelijk beginpunt voor deze opdrachten is de \href{https://github.com/HZ-HBO-ICT/ts-skeleton-app}{TypeScript Skeleton repository} van school.

\begin{enumerate}
	\item Maak een gecentreerde website met een titel ``Verlanglijstje``.
	\item Maak onder de titel een formulier met tekstvakken voor `Item` en `Link`; geef de gebruiker ook een knop `Toevoegen`.
	\item Als de gebruiker op `Toevoegen` klikt moet het item worden toegevoegd aan het lijstje.
	\item Zorg ervoor dat alle informatie aanwezig is voordat het item wordt toegevoegd, de URL moet daarnaast ook kloppen.
	\item Achter ieder item in het lijstje staan knoppen voor `Bekijken` en `Verwijderen`.
	\item Onderaan het lijstje staat een knop `Leegmaken` en `Alles Bekijken`.
	\item \textbf{Uitdaging:} voeg een knop toe om het lijstje te downloaden, dit wordt gedaan in een JSON-bestand.
\end{enumerate}

\end{document}
